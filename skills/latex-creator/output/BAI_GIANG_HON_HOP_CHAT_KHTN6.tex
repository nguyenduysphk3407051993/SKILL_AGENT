\documentclass[Main.tex]{subfiles}
\begin{document}
%%%Phần mở đầu
\chapter{Hỗn hợp chất}
\section{Chất tinh khiết, hỗn hợp, dung dịch}
\begin{Muctieu}
	\begin{itemize}
		\item Phân biệt được chất tinh khiết và hỗn hợp.
		\item Phân biệt được hỗn hợp đồng nhất và hỗn hợp không đồng nhất.
		\item Thực hiện được một số phương pháp tách chất khỏi hỗn hợp.
		\item Nêu được khái niệm về dung dịch, chất tan, và dung môi.
	\end{itemize}
\end{Muctieu}
\begin{kd}
	\immini{Trong cuộc sống hàng ngày, chúng ta thường gặp rất nhiều loại chất khác nhau. Nước khoáng đóng chai có phải là nước tinh khiết không? Tại sao trên nhãn chai thường ghi thành phần các chất khoáng như $Ca^{2+}$, $Mg^{2+}$?}{\includegraphics[width=0.5\textwidth]{images/placeholder.png}}
\end{kd}
**Phần lý thuyết**: 
\subsection{Chất tinh khiết và hỗn hợp}
\subsubsection{Chất tinh khiết}
\Noibat[\maunhan][][][]{Khái niệm}
\begin{ghinho}
	Chất tinh khiết (chất nguyên chất) là chất không lẫn chất nào khác. Chất tinh khiết có tính chất vật lý và hóa học nhất định.
\end{ghinho}
\Noibat[\maunhan][][][]{Ví dụ}
\begin{ghinho}
	Nước cất (nước tinh khiết), dây đồng (copper), khí oxygen, đường tinh luyện...
\end{ghinho}

\subsubsection{Hỗn hợp}
\Noibat[\maunhan][][][]{Khái niệm}
\begin{ghinho}
	Hỗn hợp là hai hay nhiều chất trộn lẫn với nhau. Tính chất của hỗn hợp thay đổi tùy thuộc vào thành phần và tỉ lệ các chất trong hỗn hợp.
\end{ghinho}
\Noibat[\maunhan][][][]{Phân loại hỗn hợp}
\begin{homhopdongnhat}
	Hỗn hợp đồng nhất là hỗn hợp không có ranh giới phân cách giữa các thành phần. Các chất phân bố đồng đều trong toàn bộ hỗn hợp. Ví dụ: nước muối, không khí, nước đường...
\end{homhopdongnhat}

\begin{homhopkhongdongnhat}
	Hỗn hợp không đồng nhất là hỗn hợp có ranh giới phân cách giữa các thành phần. Các chất phân bố không đồng đều. Ví dụ: hỗn hợp dầu ăn và nước, hỗn hợp cát và nước...
\end{homhopkhongdongnhat}

\subsection{Tách chất ra khỏi hỗn hợp}
\Noibat[\maunhan][][][]{Nguyên tắc}
\begin{ghinho}
    Dựa vào sự khác nhau về tính chất vật lí (kích thước hạt, nhiệt độ sôi, khả năng tan, khối lượng riêng...) của các chất trong hỗn hợp để tách chúng ra khỏi nhau.
\end{ghinho}
\Noibat[\maunhan][][][]{Một số phương pháp thường dùng}
\begin{phuongphap}
    \begin{cacbuoc}
        \item \textbf{Lọc:} Dùng để tách chất rắn không tan ra khỏi chất lỏng hoặc chất khí. Ví dụ: Tách cát ra khỏi nước bằng phễu lọc và giấy lọc.
        \item \textbf{Cô cạn:} Dùng để tách chất rắn tan được ra khỏi dung dịch (thường dùng khi chất rắn bền với nhiệt). Ví dụ: Sản xuất muối ăn từ nước biển.
        \item \textbf{Chiết:} Dùng để tách các chất lỏng không tan vào nhau (khối lượng riêng khác nhau). Ví dụ: Tách dầu ăn ra khỏi nước bằng phễu chiết.
    \end{cacbuoc}
\end{phuongphap}

\subsection{Dung dịch}
\Noibat[\maunhan][][][]{Khái niệm}
\begin{ghinho}
    Dung dịch là hỗn hợp đồng nhất của chất tan và dung môi.
    \begin{itemize}
        \item \textbf{Chất tan:} Là chất được hòa tan trong dung môi (có thể là rắn, lỏng, khí).
        \item \textbf{Dung môi:} Là chất dùng để hòa tan chất tan (thường là lỏng). Nước là dung môi phổ biến nhất.
    \end{itemize}
\end{ghinho}

%% Phần bài tập
\subsection{Bài tập}
\begin{dang}{Phân biệt chất tinh khiết và hỗn hợp}
\end{dang}
\begin{phuongphap}
    Dựa vào định nghĩa: Chất tinh khiết chỉ gồm một chất, có tính chất xác định. Hỗn hợp gồm nhiều chất trộn lẫn.
\end{phuongphap}
\Noibat[\maunhan][][\faBookmark][]{Ví dụ mẫu}
%%%%%==========VD_01==========%%%%%
\begin{vd}
	Chất nào sau đây được coi là chất tinh khiết?
	\choice
	{Nước khoáng.}
	{Không khí.}
	{\True Nước cất.}
	{Nước biển.}
	\loigiai{Nước cất là chất tinh khiết vì chỉ chứa các phân tử nước ($H_2O$). Nước khoáng, không khí, nước biển đều là hỗn hợp chứa nhiều chất khác nhau.}
\end{vd}

\Noibat[\maunhan][][\faBook][]{Bài tập tự luyện}

\phan{Trắc nghiệm nhiều lựa chọn}
%%%=============SOẠN EX Tieng Viet===============%%%
\Opensolutionfile{ansex}[Ans/LGEX-C03B15_HonHop]
\Opensolutionfile{ans}[Ans/Ans-C03B15_HonHop]

%%%%%============EX_01================%%%%%%
\begin{ex}
	Dãy các chất nào dưới đây là hỗn hợp?
	\choice
	{Nước cất, muối ăn, đường tinh luyện.}
	{\True Nước muối, không khí, nước khoáng.}
	{Đồng, sắt, nhôm.}
	{Kẽm, chì, vàng.}
	\loigiai{Nước muối (nước + muối), không khí (nhiều khí), nước khoáng (nước + khoáng chất) đều là hỗn hợp.}
\end{ex}

%%%%%============EX_02================%%%%%%
\begin{ex}
	Phương pháp nào thích hợp để tách cát ra khỏi nước?
	\choice
	{Cô cạn.}
	{\True Lọc.}
	{Chiết.}
	{Chưng cất.}
	\loigiai{Cát là chất rắn không tan trong nước, kích thước hạt lớn hơn lỗ giấy lọc nên dùng phương pháp lọc để tách.}
\end{ex}

%%%%%============EX_03================%%%%%%
\begin{ex}
	Để tách dầu ăn ra khỏi hỗn hợp dầu ăn và nước, ta dùng phương pháp nào?
	\choice
	{Lọc.}
	{Cô cạn.}
	{\True Chiết.}
	{Hòa tan.}
	\loigiai{Dầu ăn không tan trong nước và nhẹ hơn nước, tạo thành hỗn hợp không đồng nhất, phân lớp. Dùng phễu chiết để tách.}
\end{ex}

%%%%%============EX_04================%%%%%%
\begin{ex}
	Trong nước muối sinh lí (dung dịch NaCl 0,9\%), dung môi là gì?
	\choice
	{Muối ăn (NaCl).}
	{\True Nước.}
	{Nước muối.}
	{Không xác định.}
	\loigiai{Nước là chất hòa tan muối ăn, nên nước là dung môi. Muối ăn là chất tan.}
\end{ex}

%%%%%============EX_05================%%%%%%
\begin{ex}
	Hỗn hợp nào sau đây là hỗn hợp đồng nhất?
	\choice
	{Dầu ăn và nước.}
	{Cát và nước.}
	{Bột mì và nước.}
	{\True Rượu etylic và nước.}
	\loigiai{Rượu etylic tan vô hạn trong nước tạo thành dung dịch đồng nhất. Các trường hợp còn lại phân lớp hoặc lắng cặn.}
\end{ex}

\Closesolutionfile{ans}
\Closesolutionfile{ansex}

\phan{Bài tập trắc nghiệm Đúng Sai}
%%%=============SOẠN TF===============%%%
\Opensolutionfile{ansex}[Ans/LGTF-C03B15_HonHop]
\Opensolutionfile{ansbook}[Ansbook/AnsTF-C03B15_HonHop]
\Opensolutionfile{ans}[Ans/Tempt-C03B15_HonHop]

%%%%%============TF_01================%%%%%%
\begin{ex}
	Cho các phát biểu sau về chất tinh khiết và hỗn hợp:
	\choiceTF
	{\True Chất tinh khiết có thành phần hóa học và tính chất vật lí cố định.}
	{\True Hỗn hợp chứa từ hai chất trở lên trộn lẫn vào nhau.}
	{Mọi hỗn hợp đều có tính chất giống nhau bất kể tỉ lệ các chất thành phần.}
	{\True Nước cất dùng trong y tế là chất tinh khiết.}
	\loigiai{
		\begin{itemchoice}[T,T,F,T]
			\itemch Đúng. Chất tinh khiết có tính chất nhất định.
			\itemch Đúng. Khái niệm hỗn hợp.
			\itemch Sai. Tính chất của hỗn hợp thay đổi theo thành phần và tỉ lệ các chất.
			\itemch Đúng. Nước cất đã loại bỏ tạp chất.
		\end{itemchoice}
	}
\end{ex}

%%%%%============TF_02================%%%%%%
\begin{ex}
	Về phương pháp tách chất:
	\choiceTF
	{Phương pháp lọc dùng để tách chất tan ra khỏi dung dịch.}
	{\True Phương pháp cô cạn dùng để thu hồi muối ăn từ nước biển.}
	{\True Phương pháp chiết dùng để tách hai chất lỏng không tan vào nhau.}
	{Phương pháp chưng cất dựa trên sự khác nhau về nhiệt độ nóng chảy của các chất.}
	\loigiai{
		\begin{itemchoice}[F,T,T,F]
			\itemch Sai. Lọc tách chất rắn không tan.
			\itemch Đúng. Muối ăn bền với nhiệt, nước bay hơi để lại muối.
			\itemch Đúng. Dựa trên sự phân lớp của 2 chất lỏng không tan.
			\itemch Sai. Chưng cất dựa trên sự khác nhau về nhiệt độ sôi.
		\end{itemchoice}
	}
\end{ex}

%%%%%============TF_03================%%%%%%
\begin{ex}
	Trong một cốc nước đường:
	\choiceTF
	{\True Nước đường là dung dịch.}
	{Đường là dung môi.}
	{\True Nước là dung môi.}
	{\True Đây là hỗn hợp đồng nhất.}
	\loigiai{
		\begin{itemchoice}[T,F,T,T]
			\itemch Đúng. Nước đường là hỗn hợp đồng nhất của chất tan (đường) và dung môi (nước).
			\itemch Sai. Đường là chất tan.
			\itemch Đúng.
			\itemch Đúng.
		\end{itemchoice}
	}
\end{ex}

%%%%%============TF_04================%%%%%%
\begin{ex}
	Xét hỗn hợp gồm bột sắt (iron) và bột lưu huỳnh (sulfur):
	\choiceTF
	{\True Đây là hỗn hợp không đồng nhất.}
	{Có thể tách sắt ra bằng nam châm.}
	{Có thể tách sắt ra bằng cách hòa tan vào nước.}
	{\True Màu của hỗn hợp là màu xám vàng (pha trộn giữa màu của sắt và lưu huỳnh).}
	\loigiai{
		\begin{itemchoice}[T,T,F,T]
			\itemch Đúng. Hỗn hợp rắn-rắn, nhìn thấy các hạt riêng biệt.
			\itemch Đúng (nếu đề bài cho phép dùng nam châm). Sắt bị nam châm hút, lưu huỳnh thì không.
			\itemch Sai. Cả sắt và lưu huỳnh đều không tan trong nước.
			\itemch Đúng.
		\end{itemchoice}
	}
\end{ex}

%%%%%============TF_05================%%%%%%
\begin{ex}
    Về tính chất của dung dịch:
    \choiceTF
    {Dung dịch luôn ở thể lỏng.}
    {\True Dung dịch phải là hỗn hợp đồng nhất.}
    {Dung môi bắt buộc phải là nước.}
    {\True Một dung dịch có thể có nhiều chất tan.}
    \loigiai{
        \begin{itemchoice}[F,T,F,T]
            \itemch Sai. Dung dịch có thể là khí (không khí), rắn (hợp kim). Nhưng ở chương trình KHTN6 thường gặp dung dịch lỏng.
            \itemch Đúng.
            \itemch Sai. Dung môi có thể là xăng, cồn...
            \itemch Đúng. Ví dụ nước biển chứa nhiều muối tan.
        \end{itemchoice}
    }
\end{ex}

\Closesolutionfile{ans}
\Closesolutionfile{ansbook}
\Closesolutionfile{ansex}

\phan{Bài tập trả lời ngắn}
%%%=============SOẠN SA===============%%%
\Opensolutionfile{ansbth}[Ans/LGSA-C03B15_HonHop]
\Opensolutionfile{ansbt}[Ans/AnsSA-C03B15_HonHop]

%%%%%============SA_01================%%%%%%
\begin{ex}
    Không khí bao gồm các khí: nitrogen, oxygen, argon, carbon dioxide, hơi nước và một số khí khác. Không khí là chất tinh khiết hay hỗn hợp?
    \shortans{Hỗn hợp}
    \loigiai{Không khí chứa nhiều chất khí khác nhau nên là hỗn hợp.}
\end{ex}

%%%%%============SA_02================%%%%%%
\begin{ex}
    Trên bao bì của một gói bột canh có ghi các thành phần: Muối ăn, đường, bột tiêu, mì chính (bột ngọt). Gói bột canh này là chất tinh khiết hay hỗn hợp?
    \shortans{Hỗn hợp}
    \loigiai{Chứa nhiều chất khác nhau trộn lẫn nên là hỗn hợp.}
\end{ex}

%%%%%============SA_03================%%%%%%
\begin{ex}
    Khi hòa tan hoàn toàn 10 gam muối ăn vào 100 gam nước. Khối lượng dung dịch thu được là bao nhiêu gam?
    \shortans{110}
    \loigiai{$m_{dd} = m_{ct} + m_{dm} = 10 + 100 = 110$ gam.}
\end{ex}

%%%%%============SA_04================%%%%%%
\begin{ex}
    Nhiệt độ sôi của nước tinh khiết ở áp suất tiêu chuẩn là bao nhiêu độ C?
    \shortans{100}
    \loigiai{Nước tinh khiết sôi ở $100^\circ C$ (tại 1 atm).}
\end{ex}

%%%%%============SA_05================%%%%%%
\begin{ex}
    Trong phương pháp chưng cất để tách nước ngọt từ nước biển, quá trình chuyển thể nào xảy ra đầu tiên đối với nước?
    \shortans{Bay hơi}
    \loigiai{Nước bay hơi (sôi) rồi sau đó ngưng tụ lại thành nước ngọt.}
\end{ex}

\Closesolutionfile{ansbt}
\Closesolutionfile{ansbth}

\phan{Bài tập tự luận}
%%%=============SOẠN BT===============%%%
\Opensolutionfile{ansbth}[Ans/LGBT-C03B15_HonHop]
\Opensolutionfile{ansbt}[Ans/AnsBT-C03B15_HonHop]

%%%%%============BT_01================%%%%%%
\begin{bt}
    Hãy phân loại các chất/vật thể sau vào nhóm "Chất tinh khiết" hoặc "Hỗn hợp":
    Nước cất, nước chanh đường, dây điện bằng đồng, không khí, muối ăn tinh khiết, nước sông.
    \loigiai{
        - Chất tinh khiết: Nước cất, dây điện bằng đồng (copper), muối ăn tinh khiết.
        - Hỗn hợp: Nước chanh đường, không khí, nước sông.
    }
\end{bt}

%%%%%============BT_02================%%%%%%
\begin{bt}
    Em hãy đề xuất cách tách riêng từng chất ra khỏi hỗn hợp gồm: Cát, muối ăn và nước.
    \loigiai{
        Các bước thực hiện:
        \begin{enumerate}
            \item Pha hỗn hợp vào nước khuấy đều cho muối tan hết (cát không tan).
            \item Dùng phễu lọc và giấy lọc để lọc lấy cát ra khỏi dung dịch nước muối. Rửa cát trên giấy lọc bằng nước sạch và làm khô $\rightarrow$ Thu được cát.
            \item Đem phần dung dịch nước muối thu được đi cô cạn (đun nóng cho nước bay hơi hết). Nước bay hơi hết sẽ còn lại muối ăn kết tinh dưới đáy.
        \end{enumerate}
    }
\end{bt}

%%%%%============BT_03================%%%%%%
\begin{bt}
    Tại sao khi đun nước sôi để nguội, ta thường thấy có cặn trắng lắng xuống dưới đáy ấm? Điều này chứng tỏ nước tự nhiên (nước máy, nước giếng) là chất tinh khiết hay hỗn hợp?
    \loigiai{
        - Nước tự nhiên chứa các muối khoáng hòa tan (như $Ca(HCO_3)_2, Mg(HCO_3)_2...$). Khi đun sôi, các chất này bị phân hủy tạo thành kết tủa (cặn) lắng xuống.
        - Điều này chứng tỏ nước tự nhiên là hỗn hợp (dung dịch chứa muối khoáng), không phải nước tinh khiết.
    }
\end{bt}

%%%%%============BT_04================%%%%%%
\begin{bt}
    Có 3 cốc chứa 3 chất lỏng trong suốt không màu: Nước cất, nước muối, và rượu (cồn) 70 độ. Làm thế nào để phân biệt chúng bằng phương pháp vật lí đơn giản (không dùng hóa chất)?
    \loigiai{
        \begin{itemize}
            \item Cô cạn một ít chất lỏng từ mỗi cốc trên ngọn lửa đèn cồn (hoặc thìa inox đun nóng):
                \begin{itemize}
                    \item Mẫu nào bay hơi hết không để lại dấu vết gì là nước cất hoặc cồn.
                    \item Mẫu nào bay hơi để lại cặn trắng (muối) là nước muối.
                \end{itemize}
            \item Phân biệt nước cất và cồn: Ngửi mùi (cồn có mùi đặc trưng) hoặc đốt cháy (cồn cháy được, nước không cháy). (Lưu ý cồn 70 độ là hỗn hợp cồn và nước nhưng thành phần chính là cồn nên cháy được).
            \item Cách khác (nếm - chỉ dùng khi biết chắc chắn an toàn): Nước muối mặn, nước cất không vị, cồn cay/nồng.
        \end{itemize}
    }
\end{bt}

%%%%%============BT_05================%%%%%%
\begin{bt}
    Một bạn học sinh nói: "Sữa tươi là chất tinh khiết vì nó có màu trắng đồng nhất". Theo em, phát biểu đó đúng hay sai? Tại sao?
    \loigiai{
        - Phát biểu đó SAI.
        - Giải thích: Sữa tươi là một hỗn hợp gồm nước, chất béo, protein, đường lactose, vitamin và khoáng chất... Dù nhìn bằng mắt thường thấy đồng nhất (thực chất là hệ nhũ tương), nhưng nó được cấu tạo từ nhiều chất khác nhau, không phải là một chất duy nhất. Nếu để lâu hoặc dùng kính hiển vi có thể thấy các hạt cầu chất béo lơ lửng.
    }
\end{bt}

\Closesolutionfile{ansbt}
\Closesolutionfile{ansbth}

\end{document}
